\vskip1em
\textbf{A Python integrated pipeline for non-specialists, with Napari-enabled 3D exploration}

We built our pipeline in \textit{Python} as it allowed us to leverage its rich ecosystem of scientific computing libraries while keeping the code as accessible and user-friendly as possible.
We provide detailed installation instructions and \textit{Jupyter} notebooks with step-by-step instructions designed to be used by non-specialists, with a focus on ease of use and reproducibility. 
To tackle the challenges that arise from manipulating deep and dense 3D data, we provide \textit{Napari} plugins for manual registration, pre-processing, and analyzis. 
\textit{Napari} provides a user-friendly graphical interface which enables real-time visualization and exploration of 3D data during each step of the pre-processing and analyzis.
This was essential for the development of the pipeline, as the direct visual feedback allowed us to quickly iterate and optimize the processing steps, while gaining insights into the biophysical systems under study.
We believe that no other tool provides such a seamless integration of 3D data exploration and processing, making our pipeline a unique and powerful tool for the organoid research community.
The pipeline currently requires users to pre-format their raw data so that it matches given patterns (e.g the pipeline accepts appropriately-shaped \textit{tif} files, but temporal datasets can also be given as folders of \textit{tif} files to avoid fitting the whole dataset in memory).
This requirement can be cumbersome for non-specialists who may not be familiar with data formatting.
However, we plan on improving this limitation in the near future by extending the pipeline to handle a wider range of input data formats, making the pipeline more user-friendly and accessible.
While this work focuses on gastruloid datasets, we believe that the pipeline effectively extends to organoid datasets and other datasets that encounter challenges related to whole-mount deep imaging, such as tumors.


